% Options for packages loaded elsewhere
\PassOptionsToPackage{unicode}{hyperref}
\PassOptionsToPackage{hyphens}{url}
%
\documentclass[12pt]{article}
\usepackage{ebgaramond}
\usepackage{amssymb,amsmath}
\usepackage{ifxetex,ifluatex}
\usepackage{setspace}
\usepackage[top=3cm]{geometry}

\ifnum 0\ifxetex 1\fi\ifluatex 1\fi=0 % if pdftex
  \usepackage[T1]{fontenc}
  \usepackage[utf8]{inputenc}
  \usepackage{textcomp} % provide euro and other symbols
\else % if luatex or xetex
  \usepackage{unicode-math}
  \defaultfontfeatures{Scale=MatchLowercase}
  \defaultfontfeatures[\rmfamily]{Ligatures=TeX,Scale=1}
\fi

% Use upquote if available, for straight quotes in verbatim environments
\IfFileExists{upquote.sty}{\usepackage{upquote}}{}
\IfFileExists{microtype.sty}{% use microtype if available
  \usepackage[]{microtype}
  \UseMicrotypeSet[protrusion]{basicmath} % disable protrusion for tt fonts
}{}
% \makeatletter
% \@ifundefined{KOMAClassName}{% if non-KOMA class
%   \IfFileExists{parskip.sty}{%
%     \usepackage{parskip}
%   }{% else
%     \setlength{\parindent}{0pt}
%     \setlength{\parskip}{6pt plus 2pt minus 1pt}}
% }{% if KOMA class
%   \KOMAoptions{parskip=half}}
% \makeatother

\usepackage{xcolor}

\IfFileExists{xurl.sty}{\usepackage{xurl}}{} % add URL line breaks if available
\IfFileExists{bookmark.sty}{\usepackage{bookmark}}{\usepackage{hyperref}}

\hypersetup{
  % hidelinks,
  colorlinks=true,
  linkcolor=blue,
  filecolor=magenta,      
  urlcolor=cyan,
  pdfcreator={LaTeX via pandoc}}

% \urlstyle{same} % disable monospaced font for URLs
\setlength{\emergencystretch}{3em} % prevent overfull lines

\providecommand{\tightlist}{%
  \setlength{\itemsep}{0pt}\setlength{\parskip}{0pt}}
\setcounter{secnumdepth}{-\maxdimen} % remove section numbering

% \setlength{\baselineskip}{40pt}
% \setlength{\lineskip}{40pt}
% \setlength{\parskip}{10pt}
\doublespacing

\author{Prof. Lucas Silveira}
\date{Fevereiro de 2025}
\title{Supervisão e Controle de SEP\\Atividade Complementar 4}

\begin{document}

\maketitle

% \section[sec:dev-sim-mod-dnp3]{Desenvolvimento de um simulador de comunicação entre IED de proteção e SCADA-LTS utilizando os protocolos modbus-TCP e DNP3}

Com base nos dados de pontos analógicos e digitais coletados nos equipamentos presentes no pátio da SE UFC (69,0kV e 13,8kV) desenvolva um \textbf{simulador em linguagem de programação Python} que modele a comunicação em modbus/dnp3 entre os IEDs dessa subestação e seu sistema SCADA.

Para a implementação do simulador modbus utilize as seguintes bibliotecas Python:

\begin{itemize}
  \tightlist
  \item \emph{PyModbusTCP} que possui ampla documentação e está disponível no GitHub no seguinte link: \url{https://github.com/sourceperl/pyModbusTCP}
  \item \emph{DNP3-python} disponível no seguinte repositório do GitHub: \url{https://github.com/VOLTTRON/dnp3-python/tree/develop}
\end{itemize}

Considere que os seguintes IEDs têm comunicação modbus-TCP: 

\begin{itemize}
  \tightlist
  \item IED do vão de entrada de linha (Disjuntor de 69,0 kV);
  \item IEDs do vão de transformação (Disjuntor de 69,0 kV).
\end{itemize}

Considere que os seguintes IEDs têm comunicação DNP3: 

\begin{itemize}
  \tightlist
  \item IED da barra de 13,8kV (Disjuntor de 13,8 kV);
  \item IEDs de saída de alimentador (Religadores de saída de alimentador);
\end{itemize}

Os equipamentos a serem considerados de acordo com cada protocolo são:

\begin{itemize}
\tightlist
\item 01 Chave seccionadora de 69,0 kV (Vão de entrada).
\item 01 Transformador de Corrente de 69,0 kV.
\item 01 Transformador de Potencial de 69,0 kV.
\item 01 Disjuntor de 69,0 kV.
\item 02 Chave seccionadora de 69,0 kV (Vão de transformação).
\item 02 Transformadores de 69,0kV/13,8kV.
\item 02 Disjuntores de 13,8 kV (barra de 13,8kV).
\item 02 Transformador de Corrente de 13,8kV.
\item 01 Transformador de Potencial de 13,8kV.
\item 03 Religadores de 13,8 kV (saída de linha).
\end{itemize}

O sistema SCADA a ser utilizado é o software open-source \emph{SCADA-LTS} disponível no GitHub neste link: \url{https://github.com/SCADA-LTS/Scada-LTS}. Nele, cada uma das \textbf{bases de dados} devem ser configuradas e parametrizadas de acordo com o protocolo definido. Todos os pontos precisam ser listados na tela inicial do SCADA-LTS assim como um diagrama unifilar deve ser definido contendo os principais pontos de comando e supervisão dos equipamentos da subestação, proporcionando para o usuário possibilidade de visualização dos estados e facilidade para comandar os equipamentos.

\end{document}
