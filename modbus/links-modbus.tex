% Options for packages loaded elsewhere
\PassOptionsToPackage{unicode}{hyperref}
\PassOptionsToPackage{hyphens}{url}
%
\documentclass[12pt
]{article}
\usepackage{ebgaramond}
\usepackage{amssymb,amsmath}
\usepackage{ifxetex,ifluatex}

\ifnum 0\ifxetex 1\fi\ifluatex 1\fi=0 % if pdftex
  \usepackage[T1]{fontenc}
  \usepackage[utf8]{inputenc}
  \usepackage{textcomp} % provide euro and other symbols
\else % if luatex or xetex
  \usepackage{unicode-math}
  \defaultfontfeatures{Scale=MatchLowercase}
  \defaultfontfeatures[\rmfamily]{Ligatures=TeX,Scale=1}
\fi
% Use upquote if available, for straight quotes in verbatim environments
\IfFileExists{upquote.sty}{\usepackage{upquote}}{}
\IfFileExists{microtype.sty}{% use microtype if available
  \usepackage[]{microtype}
  \UseMicrotypeSet[protrusion]{basicmath} % disable protrusion for tt fonts
}{}

\makeatletter

\usepackage{parskip}
\setlength{\parindent}{0pt}
\setlength{\parskip}{15pt}


\makeatother

\usepackage{xcolor}
\IfFileExists{xurl.sty}{\usepackage{xurl}}{} % add URL line breaks if available
\IfFileExists{bookmark.sty}{\usepackage{bookmark}}{\usepackage{hyperref}}

\hypersetup{
  colorlinks=true,
  linkcolor=blue,
  urlcolor=cyan,
  pdfcreator={LaTeX via pandoc}}

% \urlstyle{same} % disable monospaced font for URLs

\setlength{\emergencystretch}{3em} % prevent overfull lines

\providecommand{\tightlist}{%
  \setlength{\itemsep}{10pt}
  \setlength{\parskip}{0pt}}
\setcounter{secnumdepth}{-\maxdimen} % remove section numbering

\title{Lista de Links sobre protocolo Modbus}
\author{Prof. Lucas Silveira}
\date{Janeiro de 2025}

\begin{document}

\maketitle

Seguem alguns links de referências e de materiais comentados em sala de
aula sobre o protocolo modbus:

\begin{itemize}
\tightlist
\item
  Especificação do Modbus: \url{https://www.modbus.org/docs/Modbus_Application_Protocol_V1_1b3.pdf};
\item
  Página em inglês na Wikipedia com explicações sobre o protocolo
  modbus: \url{https://en.wikipedia.org/wiki/Modbus};
\item
  Apresentação de slides utilizada em sala de aula sobre o protocolo
  modbus
  \href{https://drive.google.com/file/d/1dzLf-euv6dNQS82yvl6RlspNYL8PkfPy/view?usp=drive_link}{link};
\item
  Página web do software analisador de rede Wireshark
  \url{https://www.wireshark.org/};
\item
  Página web do projeto SCADA-LTS \url{http://scada-lts.org/};
\item
  Github do projeto SCADA-LTS
  \url{https://github.com/SCADA-LTS/Scada-LTS/wiki};
\item
  Página web do WSL (Windows Subsystem Linux) para instalação de uma
  distribuição Linux no Windows
  \url{https://learn.microsoft.com/pt-br/windows/wsl/install};
\item
  Página web do projeto Docker Containers utilizado para execução do
  SCADA-LTS \url{https://www.docker.com/};
\item
  Link para documentação do relé SEL 751 com capacidade de comunicação
  via protocolo modbus
  \href{https://drive.google.com/file/d/1_X_TWn8YUtYiy13mjku66pRBING1N2Ed/view?usp=drive_link}{link};
\item
  Link para documentação modbus do relé SEPAN série 80
  \href{https://drive.google.com/file/d/1mkT6fXqQvDHMV-ROjgld160GoYlrHh5H/view?usp=drive_link}{link};
\item
  Link para biblioteca Python, \texttt{PyModbusTCP} com implementação do
  protocolo modbus \url{https://pymodbustcp.readthedocs.io/en/latest/};
\end{itemize}

\end{document}
