\hypertarget{descriuxe7uxe3o-dos-procedimentos}{%
\section{Descrição dos
procedimentos}\label{descriuxe7uxe3o-dos-procedimentos}}

Utilizando a lista de equipamentos mostrada abaixo, elabore em uma
planilha computacional todos os pontos de supervisão/controle digitais e
analógicos desses dispositivos.

A planilha deve conter as seguintes colunas, nesta ordem:

\begin{itemize}
\tightlist
\item
  Equipamento DE;
\item
  Terminal do Equipamento DE;
\item
  Codificação do Cabo;
\item
  Via do Cabo;
\item
  Equipamento PARA;
\item
  Terminal do Equipamento PARA;
\item
  Página do Funcional;
\item
  Evento de Ligação;
\end{itemize}

Considere a utilização de cabos de controle com a seguinte
especificação:

\begin{itemize}
\tightlist
\item
  Para os sinais analógicos de TCs e TPs, cabos blindados de 6 vias de
  \(2,5~mm^2\);
\item
  Para os sinais digitais de supervisão e controle, cabos blindados de
  12 ou 6 vias de \(1,5~mm^2\), conforme a necessidade;
\item
  Não misturar em um mesmo cabo sinais de equipamentos distintos. Caso
  haja vias ociosas nos cabos descreva-as com a palavra RESERVA.
\end{itemize}

Incluir também na planilha os \emph{pontos de alimentação CC e CA de
cada equipamento}, conforme informações presentes nos esquemas
funcionais fornecidos em anexo. Os circuitos de alimentação CC e CA não
devem ser ligados aos painéis PPC 1 e PPC 2 e sim aos painéis de
serviços auxiliares QDCC e QDCA.

Considere a existência de dois painéis de proteção e controle (PPC 1 e
2) com a seguinte descrição:

\begin{itemize}
\tightlist
\item
  PPC 1: Contém os equipamentos de supervisão, controle e proteção do
  vão de entrada de linha de 69,0kV e dos vãos de transformação;
\item
  PPC 2: Contém os equipamentos de supervisão, controle e proteção dos
  vãos de barramento de 13,8 kV.
\end{itemize}

Para os terminais dos \emph{equipamentos DE}, considere as indicações do
diagrama funcional dos equipamentos.

Para os terminais dos \emph{equipamentos PARA} (Painéis de Proteção e
Controle - PPC1 e PPC2 ou QDCA e QDCC) considere a existência de réguas
com 40 bornes cada. Numere as réguas dentro do painel como X101, X102,
X103, etc. para o PPC1 e X201, X202, X203, etc. para o PPC2.

Os cabos com medição de tensão e corrente de TCs e TPs devem ser ligados
em chaves de aferição (S1, S2, S3, etc).

\hypertarget{lista-de-equipamentos}{%
\subsection{Lista de Equipamentos}\label{lista-de-equipamentos}}

\begin{itemize}
\tightlist
\item
  Chave Seccionadora de 72,5 kV (Entrada de Linha);
\item
  Lâmina de terra de 72,5 kV (Entrada de Linha);
\item
  Transformador de Corrente de 72,5 kV;
\item
  Transformador de Potencial de 72,5 kV;
\item
  Disjuntor de Força de 72,5 kV;
\item
  Chave Seccionadora 1 (Vão de Transformação);
\item
  Chave Seccionadora 2 (Vão de Transformação);
\item
  Transformador de Força 69/13,8kV (Vão de Transformação);
\item
  Disjuntor de Força 1 de 15,0 kV;
\item
  Disjuntor de Força 2 de 15,0 kV;
\item
  Transformador de Corrente 1 de 15,0 kV;
\item
  Transformador de Corrente 2 de 15,0 kV;
\item
  Transformador de Potencial de 15,0 kV;
\end{itemize}

\hypertarget{instruuxe7uxf5es-de-entrega}{%
\subsection{Instruções de Entrega}\label{instruuxe7uxf5es-de-entrega}}

Trabalho pode ser realizado em dupla e com data de entrega até 20/12/24.
